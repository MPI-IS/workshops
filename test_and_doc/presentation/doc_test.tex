%% Aspect ratio 16:10, delete for 4:3
\documentclass[compress,english,aspectratio=1610]{beamer}

\usepackage[english]{babel}
\usepackage[T1]{fontenc}
\usepackage[utf8]{inputenc}
\usepackage{graphicx}
\usepackage{tcolorbox}
\usepackage{ulem}
%\usepackage{pgfplots}
%\usepackage{relsize}

\setcounter{secnumdepth}{3}
\setcounter{tocdepth}{3}

%\pgfplotsset{compat=1.5, width=13cm, height=7cm}
\setbeamertemplate{caption}{\raggedright\insertcaption\par} % no "Figure X in captions"
\setbeamercovered{transparent}

    \expandafter\def\expandafter\insertshorttitle\expandafter{%
      \insertshorttitle\hfill%
      \insertframenumber\,/\,\inserttotalframenumber}

%% Evenly spread items
\let\olditem\item
\renewcommand{\item}{\setlength{\itemsep}{\fill}\olditem}

%% MPG green (Pantone 328)
\definecolor{mpg-green}{cmyk}{1,0,0.57,0.3}
\definecolor{mpg-gray}{cmyk}{0,0,0.06,0.23}

%% Symbols for footnotes
\renewcommand*{\thefootnote}{\fnsymbol{footnote}}

\newcommand{\Cpp}{C\texttt{++}}

\mode<presentation>
 {
  \usetheme{Ilmenau}
  %% Beamer color theme
  \usecolortheme[cmyk={1,0,0.57,0.3}]{structure}
  %% use MPG green also for alerted text
  \setbeamercolor{alerted text}{fg=mpg-green}
  \setbeamercolor{itemize item}{fg=mpg-gray}
  \beamertemplatenavigationsymbolsempty
  %\setbeamertemplate{footline}[frame number]
}

    \setbeamerfont{section title}{parent=title}
    \setbeamercolor{section title}{parent=titlelike}
    \defbeamertemplate{section page}{mine}[1][] {%
      \centering
        \begin{beamercolorbox}[sep=8pt,center,#1]{section title}
          \usebeamerfont{section title}\insertsection\par
        \end{beamercolorbox}
    }
%    \newcommand*{\mysectionpage}{\usebeamertemplate*{section page}}

\setbeamertemplate{section page}[mine]

\AtBeginSection[]{\subsection{}\frame{\sectionpage}}

\AtBeginSection[]
{
  \begin{frame}<beamer>
    \frametitle{Outline}
    \tableofcontents[currentsection]
  \end{frame}
}


\hypersetup{
  colorlinks,
  allcolors=.,
  urlcolor=blue,
}
%\usepackage{tikz}

%\newcommand{\bluecheck}{}%
%\DeclareRobustCommand{\greencheck}{%
%  \tikz\fill[scale=0.4, color=mpg-green]
%  (0,.35) -- (.25,0) -- (1,.7) -- (.25,.15) -- cycle;%
%}

\usepackage{amssymb}% http://ctan.org/pkg/amssymb
\usepackage{pifont}% http://ctan.org/pkg/pifont
\newcommand{\cmark}{{\color{mpg-green}\ding{51}}}%
\newcommand{\xmark}{{\color{red}\ding{55}}}%

\setbeamercovered{invisible}


\title{How to test and document your Python code}
\author{ZWE Software Workshop}
\institute{Max-Planck-Institut f\"ur Intelligente Systeme}
\date{\small{Python Introductory Workshop, Stuttgart, December 14, 2020}}
\titlegraphic{\includegraphics[height=1.5cm]{figures/minerva}}
\begin{document}
%% Title frame
\begin{frame}[plain,label=thetitle]
 \titlepage
\end{frame}

%\logo{\includegraphics[height=1cm]{figures/minerva}}


%% Table of Contents
\begin{frame}{Outline}
	\tableofcontents
\end{frame}

\section{Testing your code}
\begin{frame}{Why?}
    \begin{itemize}
        \item Write better code
        \item Save time
        \item Not look like an idiot
        \item It is fun! :)
    \end{itemize}
\end{frame}

\begin{frame}{How?}
    The standard framework for testing Python code is
    \href{https://docs.python.org/3/library/unittest.html}{unittest}.
    \begin{itemize}
        \item Create {\tt tests} packages
        \item Create modules that contains the tests with appropriate names
        \item Run for example: \newline
        {\tt \$ python -m unittest} \newline
        {\tt \$ python -m unittest test\_module1 test\_module2} \newline
        {\tt \$ python -m unittest test\_module.TestClass} \newline
        {\tt \$ python -m unittest test\_module.TestClass.test\_method} \newline
    \end{itemize}

    Alternative: \href{https://docs.pytest.org/en/stable/}{pytest}
\end{frame}

\section{Documentation your code}
\begin{frame}{Why?}
    \begin{itemize}
        \item Code usability
        \item Knowledge transfer
        \item Manage expectations
        \item It is fun! :)
    \end{itemize}
\end{frame}

\begin{frame}{How?}
	The standard framework for writing documentation for Python code is
    \href{https://www.sphinx-doc.org/en/master/}{Sphinx}.
    \begin{itemize}
        \item Create {\tt doc} folder
        \item Install sphinx with pip: \newline
        {\tt \$ pip install sphinx} \newline
        {\tt \$ pip install sphinx\_bootstrap\_theme} \newline

        \item Run: \newline
        {\tt \$ sphinx-quickstart} \newline
        {\tt \$ make html} \newline
        or \newline
        {\tt \$ sphinx-build -b html source\_dir build\_dir} \newline
    \end{itemize}
    Alternative: \href{https://www.doxygen.nl/index.html}{Doxygen}
\end{frame}



\end{document}
